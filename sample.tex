\documentclass{morelull}

\title{论谁才是乖宝宝}
\author{不愿透露姓名的王先生}
\date{\today}

\begin{document}

\maketitle

\begin{abstract}
    \vrmk{morelull}简单示例。
\end{abstract}

\tableofcontents

\section{常用格式演示}

以下列举了一些常用排版格式。

\subsection{代码}

代码直接套用\code{lstlisting},仅在外观上稍作修改。

\begin{lstlisting}[language=haskell]
    -- 这里完全就是注释。
    a :: Int
    a = 1
\end{lstlisting}

同时添加javascript支持,支持缩写成\code{js}。

\begin{lstlisting}[language=js]
    const name = "Fuck you";

    const name = 'Fuck you';
    const name = `Fuck you`;

    const inc = x => x + 1;

    // compose :: (b -> c) -> (a -> b) -> a -> c
    const compose = (f, g) => x => g(f(x));
\end{lstlisting}

\subsection{中文标点符号}

\begin{itemize}
\item{\vrmk{专名号}}
\item{\vovs{着重号}}
\end{itemize}

\section{细小提示文本}

\begin{yiys}
    这里是\vovs{引用}。
\end{yiys}

\begin{tiui}
    这里是\vovs{提示}。
\end{tiui}

\begin{tixk}
    这里是\vovs{提醒}。
\end{tixk}

\begin{vuyi}
    这里是\vovs{注意}。
\end{vuyi}

\section{阿庫西斯教义(选段)}

\begin{dkli}{教义一}{}
    阿庫西斯教徒努力就能做到,你們都是能幹的人,就算失敗了也不是你的錯,不能成功都是世界的錯。從不開心的事情中逃避即可,逃避不是失敗,有句話就叫「逃避就是勝利」。猶豫過久得出的答案,無論如何選擇都會後悔,反正都要後悔,就選擇當下的快樂吧。別太擔憂晚年如何,未來的你是否笑著,連神明都不知道,那就樂在當下吧。打到惡魔 討伐魔王!打到惡魔 討伐魔王!…
\end{dkli}

\begin{jply}{教义二}{}
    汝,迷茫的家裡蹲啊。不要太過自責……不想努力是世間的錯,本性惡劣是環境的錯,長得醜是遺傳基因的錯。不要責備自己,將責任推卸給他人便好……
\end{jply}

\begin{mkti}{教义三}{}
    汝,虔誠的教徒啊。為了不再受到惡魔所誘惑,記住這句咒語吧。「厄里斯的胸部是墊出來的」。今後若是你的心又受到誘惑,就記得詠唱這句咒語。若是遇見其他受到誘惑的人,告訴他們這句咒語也是好事一樁。
\end{mkti}

\begin{dkyi}{教义四}{}
    反正都要失敗不如放手去做,失敗之後跑路就好了
\end{dkyi}

\end{document}
